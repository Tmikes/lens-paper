\documentclass[a4paper,10pt]{article}


\usepackage[a4paper, margin=1in]{geometry}

\usepackage[usenames,dvipsnames]{color}
\newcommand{\rr}[1]{\emph{\textcolor{blue}{#1}}}

\begin{document}

\noindent\textbf{Paper: Interactive obstruction-free lensing for volumetric data visualization}\\

We have carefully read the four reviews of our submission, and we are thankful for the high level-of-detail that the reviewers took time to provide. We have covered all the
indicated points, and the current letter describes our answers to them (and, where needed, modifications done to the paper). We believe that by these changes we have managed to considerably improve the focus, quality, and positioning
of our work. To help reading, the original comments are in plain text. Our answers are in \rr{italic blue}. For brevity, we do not mention below the correction of small-scale errors such as typos and word order, nor the comments which did not require a response.\\

\noindent Sincerely,\\
The Authors\\


\section{Summary}


    1) Requirements R1 through R4 are not properly described. How does the method
    proposed in this paper meet these four requirements? Please consider each
    reviewer's comments / requests for clarification related to R1 through R4.
    
    \rr{TBD}.\\

\noindent 2) The authors need to compare their method to other F+C methods. Do these
    previous methods meet the four requirements? (one reviewer suggests including this
    information in a table).
    
    \rr{Good suggestion. We added in Sec. 2.3 a table summarizing how other F+C methods discussed in the related work address (or not) R1..R4.}\\

\noindent 3) The authors claim the method is rapid and easy to use although there are not sufficient user studies to substantiate that claim. The authors should point out the limitations of the evaluation and refrain from making unsubstantiated claims.

   \rr{Agreed. We address this point in the revision by (a) revising the claims made and refraining from making strong ones where there is no sufficient evidence; and (b) adding an user study (executed in the revision period) for one of the five use-cases, namely the baggage inspection (Sec. 5.1).}\\

\noindent 4) The reviewers point out missing references. These must be added.

    \rr{TBD}.\\

\noindent 5) Several reviewers point out typo / grammar errors. These must be fixed.

    \rr{Fixed, thank you for this type of detailed feedback (!)}

\section{Review 1}


Most important issues:
    The authors' contribution as a 'technique' is not novel. Focus+Context and lens-
    based distortion are well-studied techniques. Furthermore, the paper discusses the
    application of their 'technique' in different scenarios with different types of
    datasets, however, there lacks a comparison (pictorial or quantitative) to justify
    why their method is more effective than the already existing techniques for each
    scenario. The initial requirement of the target to be 'partially visible' for the
    technique to work weakens the uniqueness of the paper. How does the technique
    differentiate between 'target feature' and 'occluding regions', that is to say,
    how is spatial connectivity resolved?

    Strengths:
    The paper explores an interesting application of existing techniques for the
    visualization of occluded targets in volumetric datasets. Furthermore, the idea of
    interactively modifying exploration parameters, such as transfer function,
    lighting, zoom, and rotation, limited within the target lens is (to the best of my
    knowledge) new work. The attempt to recognize the requirements for occlusion
    management techniques is plausible.

    The fact that the paper focused on the claim that the technique is novel (which is
    not the case) reflects my score. However, if the authors write it as an
    application paper, I will be inclined towards accepting the paper.

  The Review

    The authors' contribution as a 'technique' (Focus+Context and lens-based
    distortion) are well-studied techniques and not novel. However, the application of
    these existing techniques for the visualization of occluded targets in volumetric
    datasets is an interesting approach. Furthermore, the idea of interactively
    modifying exploration parameters, such as transfer function, lighting, zoom, and
    rotation, limited within the target lens is (to the best of my knowledge) new
    work.

    Secondly. the paper is lacking formal justification of their work by comparing
    their results to existing methods. For example, "View-Dependent Peel-Away
    Visualization for Volumetric Data", "A Deformation Framework for Focus+Context
    Flow Visualization", "Virtual Retractor: An Interactive Data Exploration System
    Using Physically Based Deformation", "GlyphLens: View-dependent Occlusion
    Management in the Interactive Glyph Visualization", "Illustrative Deformation for
    Data Exploration", to cite but a few. The authors have done an extensive
    background study and they should have selected one state-of-the-art from each
    specific scenario and discussed how their technique is more effective than the
    state-of-the-art in section 5. The authors identify 4 requirements for occlusion
    management and therefore, one form of comparison could be how the existing methods
    lack one or several of the requirements and how the technique proposed in this
    paper fulfill all. Since the paper claims that the technique could be applied to
    general datasets (and shown in section 5), the authors need to justify how their
    technique is better than the specific scientific visualizations resolving for
    occlusions.
    A minor point: even though the authors have had discussions with experts in the
    field for each application scenario (section 5), a formal user study between
    different methods and their technique would strengthen the paper's claims.

    Revision Required:
    One major change that could really strengthen the paper would be to eliminate the
    need for the target to be 'partially visible'. If the user is able to 'poke a
    hole' into the volume and scroll through the volume to fix a target, it would
    greatly add to the paper's contribution.
    
    ALEX: Yes, but this is what we can really do, cannot we?? See the DTI example.

    - Revise the subsection "Opacity" in Sec 3.3. Why did you choose 150 points? For
    voxels having similar densities how will the voxels outside B become 'transparent'
    for $TF_global$? Needs more explanation. If you change the $TF_global$ set by the
    user, then you violate the consistency of the global context.

    - Link each subsection of Section 3 to Fig 2 for a diagrammatic explanation

\section{Review 2}

The authors proposed a technique that deforms sampling rays in a raycasting
    renderer to route the rays around occluders. The strengths of this technique is
    that it is generally applicable to various volumetric data sets, and that it
    incurs very low performance overhead over existing raycasting algorithms. The
    weaknesses would be the requirement that the target be at least partially visible,
    and that the context region is greatly disturbed such that much information is
    lost. The authors discussed both points in the text.

    Their implementation appears to work for the demonstrated use cases. The authors
    claim that the design of the lens makes ray parameters easy to manipulate, but the
    user feedback is too sparse to support their claim. In addition, there is no
    comparison with prior focus+context methods in the described use cases.

  The Review

    The authors presented a novel technique: a lens for routing sampling rays around
    occluders to reach the target. The target must be partially visible, and the
    visible part is used for initializing the lens parameters. The parameters are
    adjustable by the user in real time. Their technique is demonstrated to work with
    various types of volumetric data sets.

    The paper is easy to understand. The description of the technique is detailed
    enough to be reproducible. The figures and the supplemental video describe well
    the technical detail and the user interface.

    However, there are some important issues that remain to be addressed. First, the
    authors claim the proposed technique meets 4 requirements, but the requirements
    are not stated in enough detail. As such, it becomes confusing when prior work is
    discussed in the framework of these requirements. Second, the use case evaluation
    relies too much on insufficient user feedback. If a full user study is not
    possible, then there should be other forms of evaluation. The technical novelty is
    slightly limited due to following an existing paradigm of deforming sampling rays.
    Therefore, it is important to evaluate the system as a whole.

    Revisions required:
    1. Please clearly explain the 4 requirements R1 through R4 before evaluating prior
    work under these criteria. Specific points as follows:
    1a. It is not clear if R1 emphasizes "unobstructed" or "rapidly". Also, is it
    rapid due to a good user interface (in which case a user study might be needed to
    substantiate the claim), or is it due to computational performance?
    1b. R2 needs to be justified: Why is it desirable to be able to change *all*
    parameters?
    1c. R3 is confusing because "global" could include both focus and context,
    therefore the meaning of "global context" is not clear.
    1d. R4 can be better re-stated as handling datasets where targets and occluders
    can not be separated by transfer functions, which may depend on more than
    densities.
    2. Regarding R3 and the preservation of the context: In the proposed method, the
    user is sometimes required to peel away the context by adjusting transfer
    functions until there is a line of sight to the target, as is shown in the
    supplemental video. This does not preserve the context. There should be a
    justification on how this adjustment does not violate R3, or there should be a
    discussion in the limitations subsection.
    3. The user feedback described in section 5 is not enough to substantiate the
    claim that the poposed technique works rapidly and is easy to use. Either the
    claims need to be adjusted, or formal user studies need to be conducted.
    3a. Indeed only 3 out of 5 use cases have user feedback, and the other 2 lack
    evaluation.
    4. If space permits, the prior work and the requirements they meet should be
    organized as a table.
    5. The scattered rays intersect un-distorted rays outside of the lens. This
    potentially causes duplicate sampling. Please explain if this is an issue, and how
    it is supressed.
    6. Section 5 is too verbose for its content. Long paragraphs should be reorganized
    into shorter, more focused ones.
    7. Missing relevant prior work to which the proposed approach should be
    contrasted:
    "A curved ray camera for handling occlusions through continuous multiperspective
    visualization. Cui et al., 2010"
    "Multiperspective Focus+Context Visualization, Wu et al., 2016"
    
    

\section{Review 3}

A nice but limited contribution which is somewhat curtailed by existing work in
    the literature. There are some missing citations that need to be rectified, but
    overall, this is solid work.

  The Review

    This work proposes a focus+context technique for eliminating occlusion in a
    volumetric data visualization where the user simply defines a position, size, and
    depth for a cylindrical cutout area. The contents of the lens can be modified,
    including field of view, direction, lighting, etc, allowing for additional
    utility.

    The idea is properly motivated, with the requirements of the new technique
    elicited into four specific requirements (R1-R4). This makes a lot of sense.

    However, while the coverage of the related work generally is fair and equitable, I
    do find that some of the existing occlusion management techniques have not been
    properly covered. For example, the paper brushes X-ray techniques aside, but
    Coffin's perspective cut-away views (3DUI 2006) is quite relevant to this work, as
    is Elmqvist's dynamic transparency (INTERACT 2007). Similarly, while MoleView is
    cited, 3D explosion probes such as Sonnet's (AVI 2004) or McGuffin's (VIS 2003,
    cited) also deserves mention, particularly because this explosion idea is so
    central to this work. There is more on 3D explosion in the virtual reality
    literature, where such ideas have interactive appeal.

    Given this literature, which is relatively ample, my feeling is that this paper is
    novel solely because of its "flexible and real-time interactive modification of
    the focal point", which is a novel and powerful idea. The idea of using screen-
    space raycasting to determine the hole to push through the surface, as well as the
    selection of a visible part of the target to determine where to place the focal
    point, is interesting as well. These are relatively small contributions, but to my
    knowledge they extend the state of the art in a quantifiable way. One question
    which is not fully discussed in this paper is if it ever would make sense to
    decouple the viewpoint from the lens? At least, it seems an academically
    interesting exercise, perhaps only for demonstrating the work in the paper, but I
    didn't see it mentioned.

    Validation here is performed in five application scenarios where the prototype
    implementation is employed for different example datasets. This is reasonable and
    shows the utility of the work. Unfortunately, the companion video with the paper
    only shows two of these examples---all five would have been better.

    Overall, the paper is well written and easy to follow, with just the amount of
    necessary mathematical formalism to explain the idea. The illustrations are nice,
    as are the screenshots.

    In summary, this is a nice paper with a small but reasonable contribution---small,
    because virtually all of the ideas except the local lens controls are already
    present in the literature. Despite this limitation, I lean towards a positive
    verdict of the work, as it fills an important and useful gap in the literature.
    For this reasons, the "required changes", were the paper to be conditionally
    accepted, must be rather limited: mainly, the authors are encouraged to add the
    missing literature I mentioned above, and potentially add all five examples to the
    video.
    
    

\section{Review 4}

<b>3 - Possible Accept</b><br/> The paper is not acceptable in its current state,
    but might be made acceptable with significant revisions within the conference
    review cycle.<br/>If the specified revisions are addressed fully and effectively I
    may be able to return a score of '4 - Accept'.

  Supplemental Materials

    Acceptable

  Justification

    This paper presents a new focus+context method that allows a user to quickly
    select a focal point on a partially visible object, with the "lens" performing
    local deformation to increase visibility of that obstructed object.

    While there are small contributions to the state-of-the-art in this area, the
    authors do not focus on emphasizing that. Instead, the authors dedicate the
    majority of the paper to case studies which justify the use of a focus+context
    technique to reveal hidden structures (but without focusing on why this particular
    method is better than any others). The authors do admit that a comparative study
    is outside the scope of this paper (due to the complexity of having an
    implementation of multiple methods to work with). However, the authors do not
    perform enough basis analysis (from principals presented in the papers) to show
    the advantages of this proposed method over previous work.

    With significant revisions I think the authors could do a better job of
    emphasizing the novel aspects of their technique, in order to convince the reader
    that this version of a lens is the best option for certain visualization
    situations. With these revisions this paper would be suitable for publication.

  The Review

    This paper presents a new focus+context method for viewing hidden objects that
    works quickly after the user has selected an area of interest.

    Overall, there are interesting aspects to this method. However they are lost in
    the long, detailed case studies which don't necessarily highlight the advantages
    of this particular method (over any other focus+context method).

    In section 4 the authors indicate that they used a compositing ray function but
    that any other ray function could be used instead. I am not sure that is true -
    this wouldn't necessarily (or at least not obviously) work with something like
    minimum / maximum intensity projection.


    Revisions required:

    - The authors should significantly expand their definitions of the requirements R1
    through R4. Currently these are covered only in passing in the introduction with
    very vague definitions (what is rapid? what is easy? what are the parameters?)

    - The authors need to revisit these four requirements at some point in the paper
    to show how this newly proposed method meets all the requirements yet prior
    methods fail for one or more of these requirements.

    - Overall, the organization of the paper is a bit odd. Important concepts (such as
    the requirements) are covered only in the Introduction, and the advantages to the
    proposed method are covered in the section on Related Work (before the method is
    formally presented).  The paper organization should be revised to help clearly
    convey (and identify!) the important information in the expected sections (no one
    expects that an idea covered in just one sentence in the introduction is key to
    the paper).

    - Much of the paper is devoted to covering five case studies in detail. These are
    great - but do not clearly convey how this particular method is better than any
    other F+C method. I realize that a full comparison would require an implementation
    of several other methods and a user study (rather than just something closer to a
    demo with feedback) - I am not suggesting this (it is infeasible in the revision
    time frame). However, in each of these case studies the authors should point out
    how some novel aspect of this method was critical to the successful visualization
    of the hidden object.


\end{document}

