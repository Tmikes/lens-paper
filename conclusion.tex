\section{Conclusions}
\label{sec:conclusions}
%
In this paper, we presented a new fish-eye-like context-and-focus lens that addresses the occlusion problems inherent in scalar volume rendering. The principle of our lens consists in first gathering (squeezing) rays so that they easily pass through occluding densities (given a user-specified opacity transfer function) and next scattering (fanning out) rays to best sample the target of interest. Our lens can be directly applied to any DVR raycaster and scalar volume dataset. Its main constraint is that the user should be able to find a viewpoint from which the target of interest, deep buried in the data, is at least slightly visible. We also present several modifications of the local rendering parameters within the lens (view direction, lighting parameters, opacity transfer function) that aim to both better separate the focus (lens) from the context (volume) and also allow more detailed examining of the target. Our lens is easy to use -- all its parameters are controlled via direct mouse-and-keyboard interaction -- and can be efficiently implemented atop of a standard GPU ray caster. Our lens is especially useful for highlighting structures of interest which are both deeply embedded in volumetric data and cannot be revealed by standard transfer function manipulations due to similar densities in the occluders and target. We demonstrate these points using four use-cases involving datasets from baggage detection, fluid visualization, air traffic control, and medical imaging.

Several improvements to our proposal are possible, as follows. First and foremost, heuristics can be sought to link all our free parameters (lens size, focus depth, interpolation between focus and context) directly to the volume data, so the user interaction is minimized and therefore exploration efficiency is increased. Secondly, our lens could be extended to different types of volumetric datasets, such as multivariate (vector, tensor) fields. Last but not least, a formal wider-scale evaluation of how the lens addresses more specific tasks, and how it compares to existing tools for these tasks, is a goal we aim to pursue next.
