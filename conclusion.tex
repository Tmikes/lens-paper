\section{Conclusion}

In this paper, we detail a new deforming lens as a solution to occlusion in volume rendering. The mechanism of this interactive lens consists in firstly pushing aside occluding items in order to provide an unobstructed view, and secondly magnifying the targeted object and its local context thanks to a fish-eye field of view. The flexibility of this interactive lens allows modifying its parameters such us the rays' directions, the size of the lens, the angle of view for the purposes of adjusting an automatically provided perspective on a target, and exploring its local context. 

This lens is well suited to magnify a partially hidden object in datasets where the transfer function alone cannot resolve occlusion issues while preserving the global object structure. Three concrete scenarios using different types of volumetric datasets (baggage, streamlines, and aircraft trajectories) illustrate how to take advantage of this tool.

Some improvements can be carried out in order to automatically provide a better perspective on a selected target, such as taking into account its bounding box or automatically propose curved rays.