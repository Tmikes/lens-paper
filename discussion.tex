\section{Discussion}
\label{sec:discussion}
%
%
Several points of our lens proposal are worth discussing, as follows.

\vspace{0.15cm}
\noindent\textbf{Lens activation:} Our lens can support two types of explorations. First, when the user perceives at least a \emph{part} of a target of interest in a normal DVR image, the lens can be used to reveal that target in full detail. This is a \emph{directed} exploration, which supports the task `show me more information about \emph{this} item'. The use-cases in Secs.~\ref{sec:baggage}-\ref{sec:atc} are of this type. Secondly, the user can open up a DVR volume at a location from which no partial detail is visible. In turn, this is useful when we know that there \emph{is} an interesting target buried in the volume even without seeing it (\emph{corpus callosum} use-case in Sec.~\ref{sec:dti}), thus supports the task `show me the data I \emph{know} it is here', or for free exploration to find unknown patterns in a volume, thus supporting the use-case `show me what this volume \emph{may} hide in it'. 
In the first exploration type (target not fully occluded, no matter how small the visible fragment is), our lens is extremely simple and rapid to use -- point, click, and optionally rotate light or viewpoint. In the second exploration type (target fully occluded or not even sure whether an interesting target exists in the data), the lens is equally simple to use, but several tries for selecting a suitable focus point and lens depth may be needed. However, we argue that in the latter case, revealing such targets is more complex and effort-costly by using other visualization tools such as slices or TF parameter tuning.

\vspace{0.15cm}
\noindent\textbf{Lens shape:} Occluders are pushed away, and deformed, isotropically (Secs.~\ref{sec:scattering}, \ref{continuity}). The advantage of this simple lens model is that it requires a single parameter, the lens radius $R$, which makes its usage easy. The deformations evolve smoothly from the lens center (maximal) to outside the lens (no deformation), see Sec.~\ref{continuity}, which effectively blends the local (in lens) focus with the global (out of lens) contexts (R3). However, a side-effect is that this mechanism compresses the deformed occluders strongly close to the lens border, making them hardly visible when the lens is fully active. A possible refinement would be to reduce the deformation of the pushed-away occluders while still pushing them away, thereby improving the focus+context effect (R3). However, this would occlude the areas outside the lens with these undeformed, but pushed-away, occluders, so it will basically shift occlusion from \emph{inside} the lens to \emph{outside} and close to it. Finding an optimal balance between minimal deformation (so one can recognize the pushed-away occluders) and minimal clutter (so these occluders do not destroy the lens context) is an interesting topic for future work.

\vspace{0.15cm}
\noindent\textbf{Parameter setting:} Our lens depends on several parameters: the 2D lens center $\mathbf{f}$, lens radius $R$, lens axis direction $\mathbf{a}$, local light direction $\mathbf{l}^{lens}$, scattering start-distance $t_{min}$, and gathering and scattering parameters $\alpha$ and $\beta$. As explained in Sec.~\ref{sec:principle}, all these parameters can be easily controlled via a mouse-driven virtual trackball, key modifiers, and the arrow keys. While this seems complex at first sight, performing such operations is in fact quite simple, since the lens works at interactive frame rates (15 frames per second), so the user can quickly tune the parameters and see their effect. Moreover, all parameters start with typically good preset values (see Sec.~\ref{sec:principle} for details). A possible refinement would be to pre-segment the target of interest, based on user-given values for $\mathbf{f}$, $R$, and $t_{min}$, thereby determining $\beta$ automatically. This would help the first requirement (R1) which consists in giving a fast unobstructed view of the target. However, even if this were present, we believe that manual control of the scattering $\beta$ is important to allow users to choose their most suitable field-of-view angle. In fact, this flexibility allows a better exploration of the local context (R2).

\vspace{0.15cm}
\noindent\textbf{Implementation:} The entire lens is implemented by modifying the ray trajectories constructed in the inner loop (per-pixel raycasting) of a typical DVR raycaster\,\cite{cudasdk}. Apart from this modification, we also change the per-voxel lighting function and transfer function based on the voxel location with respect to the lens and the parameters supplied by user interaction (Sec.~\ref{sec:inter_expl}. As mentioned, such changes are limited and should be easily applicable to any raycaster. Since these changes work in a per-voxel or per-ray fashion, they are directly applicable to raycasters which parallelize computations for different ray sets. 

\vspace{0.15cm}
\noindent\textbf{Limitations:} Several limitations of our approach must be mentioned too. First, as explained, de-occluding a target requires either a small fragment thereof to be visible (in which case, de-occlusion is very simple and fast), or requires the user to choose the lens focus and target depth based on other insights than partial target visibility (which, as explained, can be done but requires such insights to be available, and more trial-and-error). At a higher level, many lens mechanisms exist in the literature, as discussed in Sec.~\ref{sec:related_work}. While we have argued that, to our knowledge, none of them simultaneously supports requirements R1,$\ldots$,R4, comparing such mechanisms with our lens for specific use-cases and datasets is still an important test for the end-to-end effectiveness of our proposal. We have not covered this point as obtaining several implementations of such lenses in the same interactive framework or tool, for comparison fairness, is very challenging. However, this remains an important open point for future work -- both for our proposal but also for all other volumetric lens proposals in the literature. Related to this, we have performed three user evaluations involving specialists in airport baggage security (Sec.~\ref{sec:baggage}), pulmonology (Sec.~\ref{sec:chest}), and air traffic control (Sec.~\ref{sec:atc}). In all cases, the users were not involved in this work, nor with other work of the authors. However, the set-up of these evaluations stays at the level of formative user experiments. To confirm and refine the obtained (positive) findings, more formal user studies are needed, which we plan to cover next.
