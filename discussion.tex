\section{Discussion}
In this paper, we presented four different scenarios where we showed how our lens is a fast and flexible way to overcome occlusion issues: the exploration of a heterogeneous data-set (baggage inspection), the analysis of a deep-buried spherical vortex (fluid flow), the visualization of a hard-to-see lung tumor , and the exploration of a special aircraft trajectory.    

Nevertheless, some aspects of the lens we presented throughout this document suffers from some limitations which we detail in the following. 


First, keeping the continuity between the inner part of the lens and the rest of the volume is very important to preserve a good understanding of object deformations. This is related to the third requirement(R3) which is to keep the global context. To ensure this continuity, we used a linear interpolation function between the final ray trajectory and the one before the previous steps. In fact, the closer a ray is to the lens border, the closer is new trajectory will be to the previous one. We used linear interpolation, whose parameter was modified with a function $ f\left(k\right) = k^2$ in the purpose of reducing the interpolation near the center of the lens. Different functions and mathematical tools could have been used but the result obtained with the current function was satisfying. 

Second, we used a circular lens throughout this study. The circular shape seemed more natural for us but, we could have proposed different lens shapes. At this stage, only the radius can be increased or reduced in order to customize this shape. Furthermore, during the deformation of the rays, the shape of the items considered as obstacles and pushed aside, are modified. It would be interesting to keep their original shapes to improve this focus + context lens (R3). A solution is to adapt our rays' trajectories modification to physical based deformation inside the lens. However keeping the original shapes of all the objects that have been pushed aside can change the global context. In fact, they will either create more occlusion outside the lens or change the items' locations outside the lens by moving them away.  


Third, the angle of view in the second step of our algorithm which allows having a good sight of the targeted item and its local context can be improved. In fact, using a segmentation algorithm that computes the bounding box of the target will help to define precisely and automatically the most suitable angle of view. This is related to the first requirement(R1) which consists in giving a fast unobstructed view on the target. However, we do think it is still important to offer the possibility to modify this angle at will. In fact, this flexibility (R2) allows further exploration of the local context.

Finally, during the first main step of our lens algorithm, we try to get closer to the targeted item or area while pushing the encountered obstacles aside. Automatically finding the right distance to the target, that offers the optimum balance between a good perspective on the full target and obstacle avoidance can be sometimes difficult according to the structure of the data-set (R1 and R4). In our future works, we will instigate automatic curved rays in deforming lens in the purposes of providing in most cases a very good perspective on the target while avoiding occlusion. 
