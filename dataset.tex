\section{From vector to volumetric dataset}


In this section, we present our rational and method to transform existing vector datasets into a volume. 
Moving objects are entities that change locations in space over time. This is the case with aircraft trajectories where an aircraft takes off at a given airport at the beginning of its strip and lands onto another one at the end of its journey. Classical methods use projected lines on a 2D visualization to explore dataset composed of such moving objects \cite{hurter2014interactive}. Even if such methods have shown valuable results, additional insight and data retrieval can be envisaged with other rendering techniques like volume rendering. While volume rendering techniques mainly focus on medial data visualization, the visualization of moving objects opens new opportunities taking advantage of the volumetric data rendering process.  Using such technique with moving object dataset is not trivial, one must transform the original data (vector data) into a compatible volumetric dataset. This process is also called rasterization which turns 3D floating points into a 3D raster grid. In the following, we explain such process and then we detail our investigated datasets.

\textbf{Dataset rasterization:} While Bresenham's line algorithm is a standard algorithm to turn a line form the vector space to a raster 2D or 3D space, it will not fulfill volume rendering compatibility. Raster line must have a given thickness to maximize the number of intercepting rays and thus ensuring their visualization thanks to volume rendering methods. Other techniques can be envisaged with the computation of meshes (i.e. pipes that encompasses the 3D line) but such computation can create geometric artifacts and might be long to process. Therefore, we used an extended version of the kernel density estimation algorithm \cite{ silverman1986density} into a 3D space. Such method has already shown interesting results with visual simplification and trajectory aggregation \cite{hurter2012graph} and is part of so-called pixel-based visualization \cite{hurter2015image}.
Such process works as follows: we first define a 3D Gaussian kernel of a radius R; the center of this kernel has the highest value, while its border the lowest. Second, we re-sample every trajectory with a minimum distance between two consecutive points inferior to the half of the kernel radius. Third, we compute the convolution of this 3D kernel with the re-sampled trajectories.

Convolution can take time with a large dataset to process, therefore, we turned this computation into the frequential space. Such technique has already been applied with trajectory dataset and can even be accelerated thanks to GPU techniques \cite{lhuillier2017ffteb}.
