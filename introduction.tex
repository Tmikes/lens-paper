Direct volume rendering (DVR) is a pervasive visualization technique for displaying 3D scalar fields with applications in engineering, material sciences, and medical imaging sciences. However widely adopted, and able to handle large datasets at interactive rates, DVR inherently suffers from the problem of \emph{occlusion}: Structures of interest located deep in the volume, called next \emph{targets}, can be hard to spot and/or explore.

To aid with this, various techniques have been designed including transfer functions, segmentation, selection, and clipping. Yet, all such techniques have limitations.  \emph{Global} mechanisms, like transfer function editing, can remove both occluders and targets if these have similar densities. In certain applications, carefully designed transfer functions exist and should be used without (significant) modifications to facilitate understanding and user training\,\cite{4276082}. \emph{Local} mechanisms like segmentation, selection, or clipping are more effective in manipulating data confined to a given spatial region. Yet, many such mechanisms assume that one can easily and accurately select targets to remove them (occluders) or keep them (occluded). This is hard to do when \emph{e.g.} one does not have direct access to the targets, or when significant 3D interaction is required to select occluder(s).

A different way to handle occlusion is to use \emph{lenses}. These are flexible lightweight tools which enable local and temporary modifications of the DVR to reveal targets while keeping the global visualization context\,\cite{595268,CGF:CGF12871,6327262}. However, efficiently selecting the target and  removing all in-between occluders is still challenging. In detail, most existing occlusion management techniques do not simultaneously meet all following requirements:
Rapidly create an unobstructed view of the target (R1), 
allow an easy change of all parameters (R2), 
keep the global context (R3), 
and handle datasets where the target and occluders have very similar densities (R4).

In this paper, we increase the flexibility of lenses for DVR exploration to jointly cover all above requirements. We propose a focus-and-context (F+C) lens that combines a distortion technique, which pushes aside the occluding objects, with a fish-eye field of view, to provide a better perspective on targets. We specifically target the use-case of \emph{partially occluded} objects, where the user has a glimpse of an interesting structure, buried deep in the data, but only slightly visible from a given viewpoint and transfer-function setting. We allow the user to `open up' the volume without changing these settings, and reveal the target, by simple point, click, and scroll operations. Next, we provide several F+C modifications of the lighting parameters, transfer function, and geometry in the focus area to better understand the target. Our technique, implemented using a CUDA-based approach, can be easily incorporated in any generic DVR system.
 
%ALEX: Removed below text, it's not really related to our contribution here
%Furthermore, performances are still a  challenge in volume rendering systems. In fact, depending on the size of the dataset and also the resolution of the resulting produced image: the rendering process can be very slow. Some optimization strategies such as empty space skipping~\cite{Liu:2009:AVR:2421899.2421919}, early ray termination~\cite{CGF:CGF12605}, multiple and adaptive resolutions allow to speed up the rendering process by increasing the frame rate. With the advent of CUDA as a higher-level GPU programming language, CUDA-based ray-casters were introduced~\cite{Kainz:2009:RCM:1661412.1618498}. 

The paper is structured as follows. Section~\ref{sec:related_work} presents related work in occlusion management, lenses, and deformations for DVR visualization. Section~\ref{sec:principle} introduces the principle of our lens. Section~\ref{sec:implem} introduces implementation details. Section~\ref{sec:scenarios} presents five application scenarios for our lens in baggage inspection, 3D fluid flow visualization, chest radiology, air traffic planning, and DTI fiber exploration. Section~\ref{sec:discussion} discusses our proposal. Finally, Section~\ref{sec:conclusions} concludes the paper.


